\subsection{New progress statement for the team profile}

For the FRR, each of us rechecked their part of the FRR and added new ideas to make the document as clear as possible, as well as to ensure that it reflects what we plan on doing. We took into consideration the feedback from both the FRR and the PDR on the path of improving the contents of the FRR.
These months we encountered some weeks in which time management was very much needed, as we had to balance our parts in this competition, as well as our work for the olympiads. But in the end, we managed, as a team, to help each other out. We also encountered problems with the practical arrangement of the components in the payload as we didn’t understand how to properly connect the components and make them fit inside the rocket. We solved those issues with the help of our parents, who work in IT and who advised us along the way. The part that challenged us the most was the mounting of the payload, as the space is very limited. 

\subsection{Tasks list}

Although the FRR is not completed, we estimate that we are able to finish the tests until we leave for the competition.

\newcommand{\done}{\textbf{\textcolor{green!50!black}{Done}}}
\newcommand{\notdone}{\textbf{\textcolor{orange!50!black}{In progress}}}

\begin{enumerate}[label=\arabic*.,leftmargin=*]
    \item Progress Report - \done
    \begin{enumerate}[label=\arabic*.]
        \item New progress statement for the team profile
        \item Tasks list
        \item Detailed project status
    \end{enumerate}
    \item Mission Overview - \done
    \begin{enumerate}[label=\arabic*.]
        \item Purpose of the mission
        \item Mission objectives       
        \item General design overview 
        \begin{enumerate}[label=\arabic*.]
            \item Structural design 
            \item Electrical design       
            \item Software design
            \item Recovery system 
            \item Ground support equipment - \done
        \end{enumerate}
    \end{enumerate}
    \item Test Plan - \done
    \begin{enumerate}[label=\arabic*.]
        \item Overview of the testing procedures - \done
        \item Test objectives and methodologies - \done
    \end{enumerate}
    \item Test Results - \notdone
    \begin{enumerate}[label=\arabic*.]
        \item Summary of test results for each component / subsystem
        \item Analysis of test data and observations
    \end{enumerate}
    \item Conclusion - \done
    \begin{enumerate}[label=\arabic*.]
        \item Summary of the FRR
        \item Recommendations for next steps
    \end{enumerate}
\end{enumerate}

\subsection{Detailed project status}

While our FRR is complete and well-worked, we have encountered some setbacks during this period. 

One thing that we are still not sure about is whether there will be interference in the radio communications between us and the rocket, as it could be caused by the speed the rocket is going or the antenna that is inside the payload. For safety, we are adding a microSD module. 

A rescale of the rocket was necessary, to ensure the payload fits tightly in the nose cone. I encountered some problems with the fins, as they seem a little too big for the rocket, however the apogee and the stability have good values.

Another problem we faced was the electronic part of the payload and, because none of us had any previous experience in this field we encountered some setbacks for example we did not really know how to connect the components and how to make sure they would remain like that during the launch and we also had some troubles finding out how to arrange them so that they would fit perfectly inside the body of the rocket. We solved those problems by having our parents close and discussing with them, as they have more experience in this field and could give us ideas and tips.

Overall, taking into account the olympiads we all had which took a portion of our time, and the problems we have faced along the way, we believe that the project came to a great finish.

To further discuss the dimensions problem, we replaced the Arduino Uno Rev3 with an Arduino Nano and found a tube with the outer diameter of 47mm, which helped us to better visualise the maximum dimensions of the payload. 

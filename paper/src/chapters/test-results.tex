We are still in the process of redoing the tests, to ensure data quality and accuracy. Some of the tests, such as the recovery test and the car test are not yet done as they are still in progress.

\subsection{Test Results}

The testing procedures outlined in the test plan were meticulously executed, and the results obtained provide valuable insights into the readiness of the CanSat payload for flight. Below are the summarized outcomes of each test:

\subsubsection{Component Testing}

All individual components underwent rigorous testing to ensure their integrity and functionality. The electrical systems performed as expected when connected separately and as a whole. Recorded values from sensors were consistent with their respective surroundings, indicating proper functionality.

\subsubsection{Drop Test}

Both drop tests were conducted successfully, with the payload remaining intact and fully functional after impact. The parachute deployment was effective in mitigating the shock, and all electrical systems continued to operate without issue. Visual inspections revealed no significant damage to the payload.

\subsubsection{Thermal Test}

During the thermal test, the payload endured exposure to elevated temperatures without adverse effects on its structural integrity or electrical components. The system remained operational throughout the duration of the experiment, demonstrating resilience to simulated overheating conditions.

\subsubsection{Vibration Test}

The vibration test subjected the payload to simulated launch vibrations, confirming the robustness of its construction. Despite exposure to mechanical stress, the payload maintained its structural integrity, and all data collection mechanisms remained operational.

% \subsubsection{Recovery Test}

% The recovery test validated the functionality of the parachute deployment system, ensuring a safe descent of the payload. Impact force measurements aligned closely with predicted values, indicating accurate performance of the recovery mechanism.

% \subsubsection{Car Test}

% In the car test, the payload successfully collected data under varying speeds and accelerations, showcasing its ability to withstand horizontal motion. All sensors provided reliable measurements throughout the duration of the experiment.

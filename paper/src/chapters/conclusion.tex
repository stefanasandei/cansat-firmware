\subsection{Summary of the FRR}

During the creation of the FRR, we established our team, split up the tasks, made plans on how the rocket was going to launch and what information it was going to gather, and selected what types of sensors and electronics we were going to use. We have bought each sensor and electronic and determined how they will all be connected, but we have yet to do that. We have also established what calculations, tables, and graphs we are going to do with the information gathered.

After all of our discussions and meetings, nearly everything needed to make this rocket a reality has been finalised. We still have to connect all the electrical components, but we do not expect to have any complications during this step. We managed to solve the problem of the dimensions by changing the Arduino Uno component to an Arduino Nano. 

\subsection{Open Points}

Some things could go wrong, but we really think that by doing great and meticulous work and checking everything as many times as needed, we will reduce that risk by a lot. We have already checked the sensors and they are working perfectly, but we will need to thoroughly check the payload when we connect all the parts so that we are sure nothing will come apart when the rocket is launched. Aside from that, by our calculations, the payload will fit very tightly inside the rocket, so we might have to adjust some things at the moment of building. We will also make sure to not forget any step of the Test Plan so as to have as little risk possibilities as possible

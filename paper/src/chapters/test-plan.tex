\subsection{Overview of the testing procedure}

The testing plan provides clear steps to ensure the payload can withstand physical, thermic and all kinds of shocks it might meet on the launch day. They view all the aspects of the payload, such as structural integrity, software functionality, parachute deployment, electrical systems. 

To ensure the quality of the rocket and the behaviour of the materials, six tests shall be conducted. Those tests are thought to analyse the structural integrity, as well as the performance of the electrical systems in the payload. [5]

\subsection{Test objectives and methodologies}

\subsubsection{Testing of the components in two phases}

\begin{itemize}
 \item Definition of Test Objectives: The payload should still be integer and fully functional, should record appropriate values of the given surroundings.
 \item Testing methodology: To ensure the parts are working and not damaged, we connect each of them, separately, to a battery. To ensure the system is working, we connect the battery în the whole system.
 \item Test Parameters: The recorded values should be plausible to the surroundings.
 \item Safety Considerations: Mounting a resistance before testing the sensors individually so they don’t burn.
 \item Justification: To ensure all the parts are properly working and assembled. This test’s objective are to ensure the parts are not damaged, to ensure they are working properly, to ensure the entire electrical system is working.
\end{itemize}

\subsubsection{Drop Test}

\begin{itemize}
 \item Definition of Test Objectives: The payload should still be integer and fully functional, should record appropriate values of the given surroundings.
 \item Testing methodology: A camera is placed so that the payload is in it’s view when it hits the ground. The rocket body is attached by the upper part of the parachute to a 1 metre-long non-stretching cord which is mounted to a solid structure, such as a ceiling, by an eyebolt strong enough to support the shock caused by the drop. The last mentioned structure is supposed to be tall enough for the rocket to not hit the ground when the parachute and the cord are extended to the maximum. Turning on the power on the electrical components. Verifying the communications and the stats. Raising the rocket to the specified height and mounting it. Releasing the system. Ensure the power wasn’t lost. Inspect the rocket for any damages to the electrical system and other parts. Verify the communications are still working.
 \item Test Parameters: The recorded values should be plausible to the surroundings.
 \item Safety Considerations: Eyes protection.
 \item Justification: Anticipates the landing. The first physical test is supposed to verify that the parachute is working properly and to ensure the components are mounted correctly. The test is predicted to generate about 50Gs of shock to the system.
\end{itemize}

\subsubsection{Drop Test}

\begin{itemize}
 \item Definition of Test Objectives: The payload should still be integer and fully functional, should record appropriate values of the given surroundings.
 \item Testing methodology: A camera is placed so that the payload is in it’s view when it hits the ground. The rocket body is attached by the upper part of the parachute to a 1 metre-long non-stretching cord which is mounted to a solid structure, such as a ceiling, by an eyebolt strong enough to support the shock caused by the drop. The last mentioned structure is supposed to be tall enough for the rocket to not hit the ground when the parachute and the cord are extended to the maximum. Turning on the power on the electrical components. Verifying the communications and the stats. Raising the rocket to the specified height and mounting it. Releasing the system. Ensure the power wasn’t lost. Inspect the rocket for any damages to the electrical system and other parts. Verify the communications are still working.
 \item Test Parameters: The recorded values should be plausible to the surroundings.
 \item Safety Considerations: Eyes protection.
 \item Justification: Anticipates the landing. The first physical test is supposed to verify that the parachute is working properly and to ensure the components are mounted correctly. The test is predicted to generate about 50Gs of shock to the system.
\end{itemize}

\subsubsection{Thermal Test}

\begin{itemize}
 \item Definition of Test Objectives: This test simulates the heating of the rocket might meet in its journey and ensures the structural integrity of the components. The temperature might loosen some contacts and glues.
 \item Testing Methodology - The test can be done by heating the rocket to 70C for 2 hours in a heating chamber, such as an oven on the Leavening function, at the specified temperature. Also, a remote thermometer is needed to measure the internal temperature of the rocket. Turn on the electrical system. Place the rocket into the oven. Turn on the oven. Modify the temperature of the oven during the 2 hours of the experiment so that the internal temperature is maintained at 65-70C. After 2 hours, turn off the oven and take out the rocket. Perform visual inspection and functional tests to ensure the rocket survived the extreme conditions and is still operating in the expected parameters. Verify the integrity of the structure and the glued components are not affected.
 \item Test Parameters: For all times, the data collected by the temperature sensor should be approximately equal to that on the oven.
 \item Safety Precautions: Fire extinguisher on hand, non-stop supervision on the electronics so they don't melt.
 \item Justification: Simulates the overheating of the components. This test is supposed to ensure the rocket can work in a hotter environment that can be caused by overheating, air friction, low thermal conductivity of the non-electric materials, etc. The parts that are analysed in this test are the electrical system and the behaviour of the materials and glues. 
\end{itemize}

\subsubsection{Vibration test}

\begin{itemize}
 \item Definition of Test Objectives: The payload should still be integer and fully functional, should record appropriate values of the given surroundings.
 \item Testing Methodology: We can use a massage gun at various intensities to simulate the movement of the rocket. Depending on the model, this can generate around 60Hz. Power on the rocket. Verify data is being transmitted. Power on the massage gun. Place the machine gun on the rocket and keep it for 1 minute in the same place, then change its position and repeat. Stop the massage gun. Inspect the rocket for functionality and structure alteration. Verify the data is still being collected. Power down the rocket.
 \item Test Parameters: Visually, the payload should not disassemble.
 \item Safety precautions: Eyes protection.
 \item Justification: Simulates the vibrations from the launch. This test simulates the turbulence the rocket might meet in its journey and ensures the structural integrity of the components. The vibrations might unscrew the screws, as well as unmount some of the components. 
\end{itemize}

\subsubsection{Recovery test}

\begin{itemize}
 \item Definition of test objectives:The payload should still be integer and fully functional, should record appropriate values of the given surroundings.
 \item Testing Methodology: The payload with the parachute attached and undeployed is held above the ground at 7-8 metres. Under that, on the ground, there is a scale placed to measure the impact force using mechanical formulas. For different heights, we will match the result given by the acceleration measured with the scale to the one predicted using Open Rocket for the real life fall.
 \item Test Parameters: Acceleration, temperature, all of the processed data, the aspect of the payload.
 \item Safety precautions: Nobody should be sitting directly under the payload.
 \item Justification: To simulate the recovery of the payload and rocket body and to ensure the parachutes are working. Similar to the Drop Test, but this one views the parachute deployment more carefully and much more precise. 
\end{itemize}

\subsubsection{Car test}

\begin{itemize}
 \item Definition of Test Objectives: The payload should still be integer and fully functional, should record appropriate values of the given surroundings.
 \item Testing Methodology: One holds the payload in one hand, sticking out of the window of a car, driving with various speeds and accelerations; This way, all the sensors, plus the 3D diagram should collect data and could be processed.
 \item Test Parameters: Acceleration, temperature, all of the processed data, the aspect of the payload.
 \item Safety precautions: The payload should be tightly held. Seatbelts should be put on.
 \item Justification: This test simulates the fall of the payload, but in a horizontal plane.
\end{itemize}
